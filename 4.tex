%%LaTeX Example file 4

%This document contains use of the multicols package
%and talks about math mode a little.

\documentclass[letterpaper]{article}

\usepackage{multicol}
\usepackage{fullpage}

\title{Columns and Math Mode:\\Acting Like a Computer Scientist\thanks{Thanks to Mark Corrigan for help writing this document.}}
\author{Jeremy Osbourne}

\begin{document}
	
	\maketitle
	
	\begin{multicols}{2}
		
		\LaTeX has a special mode for typesetting related to mathematics, called $Math Mode$. It can be turned on many ways. $Normal Math mode$ can be contrasted with $$\textrm{on its own line}$$ which typesets the segment of math mode on its own line. This can also be done with: \[\sqrt{25} = x = 5\].
		
		
		\begin{equation}
			\label{rule}
			\mathrm{VP} \rightarrow \left\{\begin{array}{c}
			\textrm{V (NP) (AP) (PP)}\\
			\textrm{VP PP}
			\end{array} \right\}
		\end{equation}
		
		We can also reference our equations, as in eq. \ref{rule}, above. Some other nifty things include:
		
		\begin{equation}
		  \label{eq:3}
		  \frac{p_{a}}{w_{a}}
		\end{equation}
		
		As well as the ability to typset multi-line equations:
		
		\begin{eqnarray}
		  10xy^2+15x^2y-5xy & = & 5\left(2xy^2+3x^2y-xy\right) \nonumber \\
		   & = & 5x\left(2y^2+3xy-y\right) \nonumber \\
		   & = & 5xy\left(2y+3x-1\right)
		\end{eqnarray}
		
	Brown for 1$^{\textrm{st}} course, white for pudding. X$_{s}$.
		
		
	\end{multicols}
	
	Finally, it is sometimes useful to remember that:
	
	\begin{equation}
		\mathrm{X''} \rightarrow \mathrm{YP} \mathrm{X}'
	\end{equation}

	
	
\end{document}